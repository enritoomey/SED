%        File: problema4_1.tex
%     Created: Wed Nov 18 10:00 AM 2015 A
% Last Change: Wed Nov 18 10:00 AM 2015 A
%
\documentclass[a4paper]{report}
\begin{document}
\begin{enumerate}
  \item For LEO, the order of magnitud of the \bold{differential} accelerations are:
    \begin{itemize}
      \item Spherical Earth = $10^0$
      \item J2 = $10^{-2}$
      \item J3 = $10^{-5}$
      \item Moon = $10^{-7}$
      \item Sun = $10^{-7}$
      \item Solar Radiation Pressure(100\%) = $10^{-8}$
      \item Solar Radiation Pressure(2\%) = $10^{-10}$
    \end{itemize}
    For GEO orbits, Moon, Sun, and Solar Pressure terms remain almost the same, and Spherical Earth, J2 and J3 terms are reduce by a magnitude of $10^{-2}$.
  \item In order to include J2 effects, we only take its long-term and secular effects. Neglecting second order effects, we can write the variation of ROE as a function of the relative excentricity and inclination vectors $\delta \vec{i}$ and $\delta \vec{e}$, and the chief's orbit inclination $i$:
    $equacion (2.28)$\\
    Integrating this equation over $u$, we get:
    $equation 2.29 & 2.30$.
    Analysing the result, we se that J2 does no affect $\delta e$ magnitud, but its phase $\phi$. The normalized speed of this variation depends mainly of the inclination $i$. Looking at the inclination vector, we see that only $i_y$ is afected by $J_2$ being its effect positive for the for $+\delta i$ and negative for $-\delta i$. Finally, looking the $\delta a, \delta \lambda$ plane, we se $J_2$ effects only $\delta\lambda$, depending on the relative inclination $\delta i$ and obsolute inclination $i$. $i$ can be chosen in order to produce an exactly oposite displacement than the one produce by Kepler due to $\delta a != 0$, in order to keep a $\delta a$ value without changing $\delta \lambda$.
  
  \item 
\end{enumerate}
\end{document}


