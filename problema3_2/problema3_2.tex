%        File: problema3_2.tex
%     Created: Tue Nov 17 05:00 PM 2015 A
% Last Change: Tue Nov 17 05:00 PM 2015 A
%
\documentclass[a4paper]{article}
\usepackage[english]{babel}
\usepackage{enumitem}
\usepackage{mathrsfs}
\begin{document}
\begin{enumerate}[label=\emph{\alph*)}]
  \item % a) Why do we use orbit elements to describe relative motion? How do we derive a linear mapping between Hill coordinates
        %    and combinations of orbit elements? What are the specific assumptions adopted?
    Using orbit elements to describe relative motion is analog to using curvilinear variables: When linearizing, the equations of motion obtained are a better aproximation to the non-linear ones. Also, for the restricted two body problem, both deputy and chief orbit elements are constant (with exception of $M$ or $f$), so their difference will also be constant. In order to map the relative orbit elements to Hill coordinates, we use the linear mapping. The deputy position vector is expresed in the chief RTN set of coordinates using the rotation matrix from chief to inertial, and from inertial to deputy. Linearizing about the chief orbits, we can express the deputy to inertial rotation matrix as:\\
    \[[ND]=[NC]+[\delta NC]\]
    The variation in the chief to inertial rotation matrix $[\delta NC]$ is expressed in terms of the relative orbit elements $\{\delta\Omega,\delta\theta, \delta i\}$. Replacing this in the expresion for the deputy position vector in the RTN chief's frame, and differentiation with respect to time, gives the linear mapping between the deputy relative position in the chief RTN frame and the diferential orbital elements.

  \item %b) Why do we use the difference in mean anomaly or mean argument of latitude rather than the differences in true anomaly
        %   or true argument of latitude to describe the relative motion?
    We use $\delta M$ instead of $\delta f$ because, for $\delta a = 0$ in unperturbed motion, its derivative is constant even for eliptic chief's orbit.
 
  \item %c) Describe the equivalent solution of TH equations in orbit element differences space. Do you recognize the periodic and
        %   secular modes? Which orbit element differences drive those modes? Do you expect the solution in orbit element differences
        %   space or YA to be more accurate? Why?
    The solution of the TH equation in relative orbit elements are the following:
    \[u(f) = \frac{\delta a}{a} - \frac{e \delta e}{2\eta^2}+\frac{\delta_u}{\eta^2} \left(\cos{f-f_u}+\frac{e}{2}\cos{2f-f_u} \right) \]
    \[v(f) = \left( \left(1+\frac{e^2}{2}\right) \frac{\delta M}{\eta^3}+\delta \omega + \cos{i\delta\Omega}\right) - \frac{\delta_u}{\eta^2}\left( 2\sin{f-f_u}+\frac{e}{2}\sin{2f-f_u}\right)\]
    \[w(f) = \delta_w\cos{\theta-\theta_w} \]
    where $u,v,w$ are non-dimensional coordinates $(x,y,z)/r$: 
    \[f_u = \tan{\frac{e\delta M}{-\eta\delta e}}^{-1}\]
    \[\theta_w = \tan{\frac{\delta i}{\sin{i \delta \Omega}}} \]
    \[\delta_u = \sqrt{\frac{e^2\delta M^2}{\eta^2}+\delta e^2} \]
    \[\delta_w = \sqrt{\delta i^2+\sin{i}^2\delta\Omega} \]
    \[\eta = \sqrt{1-e^2} \]

    Secular efects are hidden in $\delta M$, which grows for non-zero differences in $a$:
    \[\delta M = \delta M_0 - \frac{3}{2}\frac{\delta a}{a}(M-M_0)\]

    This solution is more accurate than the YA solution because $\delta a$ is not approximated by the linearization process, thus bounded orbits can be more accurately design.

  \item % d) What is the key difference between the solution of the equations of relative motion in orbit element differences space for
        %    arbitrary and small eccentricity?
    When assuming $e\rightarrow 0$, $r$ becomes a constant, so we can expres the solution for $(x,y,z)$ instead of $(u,v,w)$. This has the advantage that the equations now gives true separation, not affected by the position in the chief's orbit. Also, the solution can be easily compared with the amplitud/phase solution of the HCW equation, showing the direct relation between HCW integration constants and the relative orbit elements.

  \item % e) Describe the linear mapping provided by Schaub between Hill coordinates and orbit element differences for near-circular
        %    orbits. How does it compare to the solution of the HCW in amplitude/phase form?
    For near-circular orbits ($e<\rho/r\ll1$, excentricity smaller than relative separation) terms containing the eccentricity can be dropped, and the following aproximations can be made: $r\approx a, ~\eta \approx 1,~f_x = f_y \approx 0,~f\approx M=n\cdot t$. Comparing the solution obtain to the HCW solution in phase form:
    \begin{itemize}
      \item $x(t)$ equation has an offset proportional to $\delta a$ not present in the HCW solution.
      \item The $x,y$ solution has the same eliptical shape in both cases ($y$ axis doubles $x$ axis), with its mayor axis equal to $2a\delta e$.
      \item $y(t)$ solution has a secular term for $\delta a \neq 0$ in $\delta M$.
    \end{itemize}

  \item %f ) What is the typical range of validity of the linearized equations of relative motion? What is the order of magnitude of the prediction errors over 1 orbit?
    The range of validity of the linearized equations of relative motion depends on the relative separation of the spacecrafts. For $\rho/r\approx 0.003$ and $a=7555~km$, we can expect an error in the solution of arround 40 m error. The error can increase up to 500 m in the case of the near-circular orbit solution for $e=0.13$.

  \item % g) What are the formal differences between the orbit element differences of Schaub and the relative orbit elements (ROE) of    D’Amico? Why do we introduce the relative eccentricity vector and the relative mean argument of latitude?
    D'Amico's ROE use $\lambda,~\delta \vec{e},~\delta \vec{i}$ intead of $\omega,~\Omega,~M,~e,~i$. Only $\delta a$ is mantain. Te relationship between both ser of parameters is given by:
    \[\delta\lambda=(M_d-M)+(\omega_d-\omega)+(\Omega_d-\Omega)\cos{i} \]
    \[\delta\vec{e} = \left( \begin{array}{cc} e_d\cos{\omega_d}-e\cdot\cos{\omega} \\
      e_d\sin{\omega_d}-e\cdot\sin{\omega} \end{array} \right) \]
    \[\delta\vec{i} = \left( \begin{array}{cc} i_d-i \\ (\Omega_d-\Omega)\sin{i} \end{array} \right) \]
    The relative excentricity vector $\delta\vec{e}$ aboids indeterminations for circular orbits, where $e=0$ and $\omega$ could have any value. In the same way, the relative excentricity vector $\delta\vec{i}$ solves the intedertimation for equatorial orbits, where $i=0$ and $\Omega$ could have any value.

  \item % h) Describe the linear mapping between Hill coordinates and ROE and compare with the solution of HCW and Schaub  mapping.
    In the radial direction, linear mapping between Hill coordinates and ROE gives the same result that HWC linear solution except for the offset component $\delta a$. In the along-track direction the amplitud of the periodic oscilations doubles the radial motion, as well as in HCW, but there is also a secular effect due to $\delta\lambda$ and $\delta a$. For the cross track direction the solution is depends only on relative $i$ and $\Omega$, and is independent of the $x,y$ solution, as wel as in HCW.

  \item % i) How do we achieve bounded and centered relative motion in ROE space? Show the analytical expression for the minimum separation perpendicular to the flight direction as a function of ROE. Which choice of ROE guarantees a minimum separation perpendicular to the flight direction at all times? Which choice of ROE maximizes this minimum separation?
    To obtain a bounded motion, $\delta a=0$ must be satisfy. To have a centered relative motion, once the previous condition satisfy, we need $\delta \lambda = 0$, which means $\delta(\omega+M)=-\delta\Omega\cos{i}$. In order to guarantee the minimal separation perpendicular to the flight direction, we must look the the $r,n$ motion:
    \[ r_{rn}(u) = a\cdot \sqrt{{\delta r_r(u)}^2+{\delta r_n(u)}^2} = \delta a\cdot sqrt{{-\delta e_x\cos{u}-\delta e_y\sin{u}}^2+{-\delta e_x\cos{u}-\delta e_y\sin{u}}^2} \]
    \[\rightarrow ~\delta r_{min}^{nr} = \frac{\sqrt{2}a|\delta e \cdot \delta i|}{\sqrt{\delta e^2+\delta i^2+|\delta e + \delta i| \cdot |\delta e - \delta i|}} \]
  Choosing $\delta e~//~\delta i$ maximize the minimim distance between spacecraft normal to the trayectory.

  \item % j) What are the relative perigee and relative ascending nodes? Sketch to show their meaning. Where are they located w.r.t. the absolute orbit for parallel relative eccentricity/inclination vectors? Is the relative motion clock-wise or counter-clock-wise in the TR and NR planes, explain?
    The relative perigee is the phase of the relative excentricity vector $\varphi = Arg[\delta \vec{e}] = \tan^{-1}{\frac{\delta e_y}{\delta e_x}}$, and the relative ascending node is the argument of the relative inclination vector $\vartheta = Arg[\delta \vec{i}]= \tan^{-1}{\frac{\delta i_y}{\delta i_x}}$. This angles are related with the absolute orbit mean argument of latitud in the following way: when $u = \varphi$ the maximum radial separation occurs; when $u = \varphi+\pi$ the maximum along-track separation occurs; and when $u = \vartheta+\pi$ the maximum cross-track separation occurs. The relative motion in the TR plane is counter-clockwise for $\delta e > 0 $. The same is valid for the RN frame for $\delta e > 0$ and $\delta i > 0$.

  \item % k) How can we define safe and unsafe formations, rationale? Can we obtain smaller baselines in the TR plane by a proper selection of the angle enclosed between relative eccentricity and relative inclination vectors? Show a numerical example.


\end{enumerate}
\end{document}

