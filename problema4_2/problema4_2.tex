%        File: problema4_2.tex
%     Created: Wed Dec 09 02:00 PM 2015 A
% Last Change: Wed Dec 09 02:00 PM 2015 A
%
\documentclass[a4paper]{article}
\usepackage[english]{babel}
\usepackage{enumitem}
\usepackage{mathrsfs}

\begin{document}
\begin{enumerate}[label=\emph{\alph*)}]
  \item % a) Describe the classical formation-flying configurations in LEO (ATO, ITO, GCO, PCO) through ROE? 

  \item % b) What is the interferometric baseline for digital elevation modeling through synthetic aperture radar? How can we best (low cost, low risk) build such a baseline and modify it depending on the terrain under observation?

  \item % c) How do the ROE change during formation maintenance when using a double-impulse in-plane and single-impulse out-of-plane maneuvering scheme? Sketch geometry during multiple maneuver cycles
    
  \item % d) How do we select the control windows and/or maneuver cycle for in-plane and out-of-plane control?

  \item % e) How much does it cost to keep a relative eccentricity/inclination vector separation with nominal relative perigee and relative ascending node at 90deg mean argument of latitude? Assume length of nominal relative eccentricity/inclination vectors at 300m, a polar orbit, and a daily maneuver cycle.

  \item % f) What are the pros and cons of an autonomous vs ground-in-the-loop control architecture?

  \item % g) Consider an eccentric chief orbit, what is the size and location of the minimum-fuel maneuver to establish bounded relative motion, rationale?


\end{document}


