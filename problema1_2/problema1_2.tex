\documentclass[a4paper]{article}
\usepackage[english]{babel}
\usepackage{enumitem}
\usepackage{mathrsfs}
\begin{document}
\begin{enumerate}[label=\emph{\alph*)}]
  \item
    \begin{itemize}[label=\textbullet]
      \item ECI:
	\begin{itemize}
	  \item Plane: Equatorial plane.
	  \item Origin: earth center of mass.
	  \item Axes pointing: x vernal equinox, z earth rotation axes.
	\end{itemize}
      \item Perifocal:
	\begin{itemize}
	  \item Plane: orbit plane
	  \item Origin: orbit focal point
	  \item Axes pointing: x focal to periasis, z normal to orbit.
	\end{itemize}
      \item ECEF:
	\begin{itemize}
	  \item plane: ecliptic plane 
	  \item origin: earth center of mass.
	  \item axes pointing: x intersection between greenwitch latitud  and equator plane
	\end{itemize}
      \item RTN:
	\begin{itemize}
	  \item plane: orbit plane
	  \item origin: satellite
	  \item axes pointing: x = r(position), y = v(velocity)
	\end{itemize}
      \item Polar rotation coordinates:
	\begin{itemize}
	  \item plane: ecliptic plane
	  \item origin:
	  \item axes pointing:
	\end{itemize}
    \end{itemize}
  \item The assumptions behind the restricted earth two body problem are that the secondary mass is negligible compared to m1, and that m1 has a perctly spherical gravitational field. There's no external forces.
  \item Fundamental orbital differential equations:
    \[\ddot{\vec{r}}+\mu\frac{\vec{r}}{r^3}=0\]
    Polar coordinate solution:
    \[r = \frac{p}{1+e\cdot\cos{f}} \qquad \textrm{with:}\]
    \[p = \frac{h^2}{\mu},~~ e=\sqrt{1+\frac{2e h^2}{\mu^2}} ~~ \textrm{and} ~~ f=\theta-\omega\]
    
  \item To derive the solution in inertial coordinates first we construct the position vector in the perifocal axis:
    \[[r]_{\mathscr{P}}=[r\cos{f};r\sin{f};0] = []\]
  Then, we use the three orbit parameters that define the orbit in the inertial frame to rotate the position vector define in the pericoal axis to the inertial axis:
 % \[[r]_{\mathscr{I}} = {T}_{\mathscr{P}}^{\mathscr{I}}~(\omega,i,\Omega) ~ [r]_{\mathscr{P}\]
  
  \item (answer afterwards)
    
  \item The true anomaly $f$  or the mean anomaly $M$
    
  \item Angular momentum and energy.
    
  \item Yes, for certain orbits the solutions becomes singular (for example, ciruclar orbits -e=0-, or equatorial orbits)
   
  \item
    \[\vec{r} = 2 \hat{i}\]
    \[\vec{v} = \frac{-1}{\sqrt{2}}+\frac{1}{\sqrt{2}}\hat{j}\]
    \[\vec{h} = \vec{r} \times \vec{v}= \frac{2}{\sqrt{2}}\hat{k} \rightarrow ~ h ~ \textrm{is aligned with Earth's rotating axis, so the orbit is equatorial.}\]
    To evaluate if the orbit is circular, elliptical, parabolic or hyperbolic we check the value of $E$:
    \[E = -\frac{\mu}{2a} = \frac{v^2}{2}-\frac{\mu}{r} = \frac{1}{2} - \frac{1}{2} = 0 \quad \rightarrow \quad \textrm{The orbit is parabolic.}\]

\end{enumerate}

\end{document}

