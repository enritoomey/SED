%        File: problema5_1.tex
%     Created: Fri Nov 20 10:00 AM 2015 A
% Last Change: Fri Nov 20 10:00 AM 2015 A
%
\documentclass[a4paper]{article}
\usepackage[english,spanish]{babel}
\usepackage{enumitem}

\begin{document}
\begin{enumerate}
  \item The key metrology systems used for spacecraft formation-flying and rendezvous are:
    \begin{itemize}
      \item Gyrometer Assembly: provides angular rates for attitude determination.
      \item Accelerometer Assembly: monitors ATV boosts.
      \item Two Star Sensors for absolute attitude information.
      \item Two telegoniometers: lasers with two mirrors for beam steering.
      \item Two vidiometers.
      \item GPS/GNSS.
      \item Radio frequency:provides GPS like signals for relative navigation, applicable for deep space navigation.
      \item Vision-based systems.
    \end{itemize}

  \item We could need relative navigations for any mision involving two spacecraft, independently of the existance of a automatic control system. We could use relative navigation to improbe orbit determination and better plan future manouvers (no need to be automatize).

  \item The two basic methods for relative/absolute orbit determination are:
    \begin{itemize}
      \item Sequential estimation, where a new estimate of the state vector is obtained after each obsevation, suited for on board implementation. This method converge more quickly at the cost of stability. Also, it is more sensitive to bad lectures.
      \item Batch estimation, where all observations are processed and combined to obtain a single update state vector. This method is best suited for an on-ground implementations, where the state vector can be less frequently updated.
    \end{itemize}
  
  \item Deterministic methods are not use because they are no able to account for uncertainties and include more state parameters.(But why?)

  \item The most common filter used for on-board navigation is the Kalman Filter, which relays on the assumption that meassurements are unbiased and timewise uncorrelated (meassurement error in time $T+1$ is not affected by meassurement in time $T$. Time updates of both the state vector and covariance matices is done using a propagation matrix $\Phi$ based in the linearization of the dinamic equations. $\Phi$ is calculated based on the state values of the previos time. Measurement update are done comparing new measurements with estimated measurements based on previos measurements propagated through time. The merge between propagated measurements and new measurements is done using the gain matrix $K$, calculated from the correlation matrix propagated, the uncertainty matrix of the meassurements, and the partial derivative of the measurement to state vector matrix $H$. The key ingrediants of this filter are:
    \begin{itemize}
      \item Dynamical model.
      \item Measurement model.
      \item Uncertainty state estimate.
      \item Uncertainty of measurements.
      \item Uncertainty of dynamics.
      \item Loss function to be minimized.
    \end{itemize}

  \item The main change in the filter due to a change in the measurements type would be in the measurement model $h$ and the partial derivatives of the measurements with respect to the state vector, $H$. For example:
    
  \item The goal of TAFF system is to determine the micro-thruster commands in order to keep correct fly formation in the orbit intrack-radial plane (orbit plane) using navegation data from GNSS, and to provide the rest of the system predicted minimim separation and relative position and velocity. TAFF also handles ground planned maneuvers for absolute orbit control, which are used to better propagate meassurements, and to assure no overlaping between both controls.
  \item The state representation for the relative motion used are the relative osculating orbit parameters (in fact, due to the close position of both spacecraft, we can assume without much error that mean and osc. relative orbit elements are similar).
    $write state representation vector$\\
    The observer measures relative position and velocity (measured and express in the ECEF frame). The perturbations taken into account are $J_2$ secular perturbations, throuth the $3\gamma\sin{i}^2$, $-12\gamma \sin{2i}$ and $\dot{\phi}$ terms in the transition matrix $\Phi=\frac{\partial \Delta \alpha (t)}{\partial \Delta \alpha (t_0)}$.

  \item In order to compute the relative position and velocity in the RTN frame directly from the ECEF data is:
\end{enumerate}
\end{document}


