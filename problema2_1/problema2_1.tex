/*set filetype:tex*/
\begin{document}<++>
a) There are two aproaches: asuming an harmonical potential perturbation force (Lagrange Planetary Equations), or directly integrating the orbit parameters variations due to arbitrary perturbations forces expressed in the RTN frame (Gauss variational equations).

b) The method of variation of parameters cosist on the following steps:
	-Take the homogeneous solution of the differential equation.
	- Take the integration constants of the 'unperturbed' motion.
	- Express these constans as a function of time.
	- Introduce ad-hoc constrains to simplify equations.
	- Solve for new, now time-dependent parametes.

c) Both sets of equation uses the oculating constrains in order to obstain the three extra degrees to solve the VOP problem:
$\frac{\partial f}{\partial oe} \dot{oe} = 0$
This means that the trajectory in the inertial configuration space is always tangential to an 'instantaneous' elipse (or hyperbola) defined by the instantaneous values of the time-varying orbital elements.

 Lagrange variational equations:
	In presence of a conservative, position-only dependent perturbing potential, VOP (variation-of-parameters) leads to the LPE. LPE uses geo-potential expansion in spherical harmonics to model earth gravitational field.
	$\frac{d\vec{oe}}{dt} = \L^{-1}\frac{\delta R}{\delta \vec{oe}} $
  Gauss variational equations:
	Gauss variational equation does no assumption on $\vec{d}$
	$\dot{oe}=\frac{\partial oe}{partial v} d$

d) 

