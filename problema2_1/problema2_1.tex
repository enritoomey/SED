\documentclass[a4paper]{article}
\usepackage[english]{babel}
\usepackage{enumitem}
\usepackage{mathrsfs}
\usepackage{amsmath}
\newcommand\norm[1]{\left\lVert#1\right\rVert}
\usepackage{graphicx}
\graphicspath{ {images/} }
\usepackage[makeroom]{cancel}

\begin{document}
\title{Questions 2.1}
\author{Enrique C. Toomey}
\date{\today}
\maketitle
\begin{enumerate}[label=\emph{\alph*)}]
  \item %a) How do we solve the fundamental differential equations in the presence of perturbations?
    There are two aproaches: asuming an harmonical potential perturbation force (Lagrange Planetary Equations), or directly integrating the orbit parameters variations due to arbitrary perturbations forces expressed in the RTN frame (Gauss variational equations).
  \item %b)  we expect to only spend a tiny fraction of this in the next few years.
    The method of variation of parameters cosist on the following steps:
    \begin{itemize}
      \item Take the homogeneous solution of the differential equation.
      \item Take the integration constants of the 'unperturbed' motion.
      \item Express these constans as a function of time.
      \item Introduce ad-hoc constrains to simplify equations.
      \item Solve for new, now time-dependent parametes.
    \end{itemize}
  \item % c) What are the Lagrange’s and Gauss variational equations? Under which assumptions are they derived? What are the main differences?
    Both sets of equation uses the oculating constrains in order to obstain the three extra degrees to solve the VOP problem:
    \[\frac{\partial f}{\partial oe} \dot{oe} = 0\]
    This means that the trajectory in the inertial configuration space is always tangential to an 'instantaneous' elipse (or hyperbola) defined by the instantaneous values of the time-varying orbital elements.
    \begin{itemize}
      \item Lagrange variational equations: In presence of a conservative, position-only dependent perturbing potential, VOP (variation-of-parameters) leads to the LPE. LPE uses geo-potential expansion in spherical harmonics to model earth gravitational field.
	\[\frac{d\vec{oe}}{dt} = \mathscr{L}^{-1}\frac{\delta R}{\delta \vec{oe}} \]
      \item Gauss variational equations: does no assumption on $\vec{d}$
	\[\dot{oe}=\frac{\partial oe}{partial v} d\]
    \end{itemize}
  
  \item % d) How can we incorporate higher order geopotential coefficients in our solution? How can we incorporate atmospheric drag in our solution? How can we incorporate impulsive maneuvers in our solution?
    Perturbation forces $d$ due to the geopotential $R$ are computed through:
    \[ d = \frac{\partial R}{\partial r}\]
    where $R$ can be modeled as a series of Legendre polynomials $P_k$:
    \[ R = -\frac{\mu}{r} \sum_{k=2}^{n}{J_k \left( \frac{Re}{r} \right)^k P_k(\cos{\phi})}\]
    where $\cos{phi}=\sin{i}\cdot\sin{f+w}$ and $P_k(x) = \frac{1}{2^k !k}\frac{d^k}{dx^k}\left[(x^2-1)^k\right]$. The larger $n$ is (number of terms in the series), the higher the order of the potential model used is, leading to more realistic earth modelling.\\
    Atmospheric drag can be incorporated using gauss variational equations, with osculating constrains. Acceleration $d$ due to drag is calculated through:
    \[ a_D = -\left(\frac{A}{m} \right) C_D \rho \frac{v^2}{2}~\hat{i}_v \]
    As th versor $\hat{i}_v$ is not aligned with the RTN frame, the following rotation matrices may be use:
    \[ \left( \begin{array}{c} a_r \\ a_t \end{array} \right) = \frac{1}{\sqrt{1+e^2+2e\cos{f}}} \left[ \begin{array}{c c} 1+e\cos{f} & e\sin{f} \\ -e\sin{f}  & 1+e\cos{f} \end{array} \right]  \left( \begin{array}{c} a_{D\perp} \\  a_D \end{array} \right) \]
    Maneuver can be modeled as instantaneous variations of velocity, which cause instantaneous variations in the relative orbit elements.
  \item % e) Describe the key effects of J2, drag, and impulsive maneuvers on the orbit solution
    J2 effects can be decompose in short term and long term oscilations, and secular effects. Secular effects affects $\Omega$, $\omega$ and $M_0$ in the following way:
    \begin{minipage}{\textwidth}
      \begin{flushleft}
	\[ \frac{d\bar{\Omega}}{dt} = -\frac{3}{2} J_2 \left( \frac{Re}{\bar{p}} \right)^2 \bar{n} \cos{\bar{i}} \]
	\[ \frac{d\bar{\omega}}{dt} = \frac{3}{4} J_2 \left( \frac{Re}{\bar{p}}\right)^2  \bar{n} (5\cos{\bar{i}}^2 -1) \]
	\[ \frac{d\bar{M_0}}{dt} = \frac{3}{4} J_2 \left( \frac{Re}{\bar{p}}\right)^2  \bar{n} \sqrt{1-\bar{e}^2} (3\cos{\bar{i}}^2 -1) \]
      \end{flushleft}
    \end{minipage}

    Atmospheric drag's main effect is the semi-mayor axis $\bar{a}$  and excentricity $\bar{e}$ reduction (orbits tend to be circular). It also introduce periodic effects on $\omega$, $M$ and $e$.\\
    Maneuvers can affect any of the six orbital elements:
    \begin{itemize}
      \item Cross-track maneuvers affects $(\bar{i},\bar{\Omega},\bar{\omega})$.
      \item Along-track maneuvers affects $(\bar{a},\bar{e},\bar{\omega},\bar{M_0})$.
      \item Radial maneuvers affects $(\bar{a},\bar{e},\bar{\omega},\bar{M_0})$, though less strongly than along-track maneuvers.
    \end{itemize}

  \item % f) What are the inclinations which remove secular effects of J2 on the orbit elements?
    For polar orbits $(i=90^o)$, $d\bar{\Omega} / dt$ effects disapear. To remove $\bar{\omega}$ and $\bar{M_0}$ effects, $i$ must be equal to $\cos^{-1}{\sqrt{1/5}}$ and $\cos^{-1}{\sqrt{1/3}}$ respectively.
 
  \item % g) What are the secular effects of atmospheric drag on the orbit elements?
    Secular effects of atmospheric drag are:
    \[ \frac{d\bar{e}}{dt} = -\left(\frac{A}{m} \right) C_D \rho \bar{e}; \qquad  \frac{d\bar{a}}{dt} = -\left(\frac{A}{m} \right) C_D \rho \frac{v^3}{\bar{a}\bar{n}^2} \]

  \item % h) Do out-of-plane impulsive maneuvers affect the in-plane orbit motion? If yes, is this conclusion still valid if we use the so-called longitude λ = ω + M0 + Ωcos(i) to describe the phasing of the spacecraft along its orbit?
    Yes, out-of-plane impulse maneuvers affect $\omega$ in the following way:
    \[\Delta\omega = -\frac{\Delta V_h r \sin{f+\omega}\cos{i}}{h\sin{i}} \]
    In the case we use $\lambda = \omega+M+\Omega\cos{i}$:
    \[\arraycolsep=1.4pt\def\arraystretch{2.2}
    \begin{array}{c c l} \Delta\lambda|_{\Delta v_h} &=& \Delta \omega|_{\Delta v_h} + \cancelto{0}{\Delta M|_{\Delta v_h}} + \Delta \Omega|_{\Delta v_h} \cos{i} - \Omega \sin{i} \Delta i|_{\Delta v_h} \\
	&=& \cancel{-\frac{\Delta V_h r \sin{f+\omega}\cos{i}}{h\sin{i}}}+ \cancel{\frac{\Delta V_h r \sin{f+\omega}\cos{i}}{h\sin{i}} } - \frac{\Omega \sin{i}\cos{f+\omega}r\Delta v_h}{h}  \\
    \Delta\lambda|_{\Delta v_h} &=& - \frac{\Omega \sin{i}\cos{f+\omega}r\Delta v_h}{h} \end{array}\] 

  \item % i) Where would you place a maneuver to change inclination only, how oriented in RTN? Where would you place a maneuver to change right ascension of ascending node only, how oriented in RTN? Where would you place a maneuver to change the semi-major axis, how oriented in RTN and w.r.t velocity?
    To change $i$ a cross-track maneuver in needed in the line of node $\theta = f + \omega = 0$. To change $\Omega$, the cross-track maneuver must be made $90^o$ from the line of nodes $(\omega+f=90^o$, not valid for polar orbits). To change the semi-mayor acis the maneuver must be made in the direction of the velocity vector (this may affect $e$, $\omega$ and $M_0$ also, but would not alter $i$ and $\omega$). In the RTN frame the maneuver is:
    \[a_R = \frac{e\sin{f}}{\sqrt{1+e^2+2e\cos{f}}} a_v \]
    \[a_N = \frac{1+ e\cos{f}}{\sqrt{1+e^2+2e\cos{f}}} a_v \]

\end{enumerate}
\end{document}

